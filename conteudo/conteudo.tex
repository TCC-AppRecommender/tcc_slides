\section{Contextualização} % (fold)
\label{sec:contextualizacao}

% section what_is_solid_ (end)

\section{Problemática} % (fold)
\label{sec:problematica}

\begin{frame}
	\say{\textit{Every class should have responsibility over a single part of the functionality
  provided by the software, and that responsibility should be entirely encapsulated by the
  class.}}, \\DeMarco, Tom
\end{frame}

\begin{frame}
  \begin{figure}[p]
        \centering
        \includegraphics[width=0.8\textwidth]{doxyparse_pattern.png}
        \label{fig:Pattern}
    \end{figure}
\end{frame}

\begin{frame}
  \begin{figure}[p]
        \centering
        \includegraphics[width=0.8\textwidth]{clang_antipattern.png}
        \label{fig:antiPattern}
    \end{figure}
\end{frame}

% section s (end)

\section{Objetivo} % (fold)
\label{sec:objetivo}

\begin{frame}
	\say{\textit{Software entities should be open for extension, but closed for modification.}}, \\Meyer, Bertrand
\end{frame}

\begin{frame}


\begin{Parallel}[v]{5cm}{5cm}
    \ParallelLText%
    {
    	\textbf{Bertrand Meyer, 1988}

    	The idea was that once completed, the implementation of a class could only be modified to correct errors.
    }
    \ParallelRText%
    {
    	\textbf{1990s}

    	The implementations can be changed and multiple implementations could be created and polymorphically substituted for each other.
	}
\end{Parallel}

\end{frame}

\begin{frame}

\begin{figure}[h!]
  % \caption{The Extractor inherited.}
  \centering
    \includegraphics[width=0.5\textwidth]{conteudo/Extractor}
    \caption{Extractor inherited}
\end{figure}
    
\end{frame}
% subsection use_in_analizo (end)


% section o (end)

\section{Sistemas de recomendação} % (fold)
\label{sec:sistemas_recomendacao}
\begin{frame}
    \say {\textit{Sistemas de recomendação são ferramentas computacionais e
técnicas usadas para produzir recomendação de itens úteis a um
usuário}},\\(MAHMOOD; RICCI, 2009)
\end{frame}

\begin{frame}

\begin{figure}[h!]
  \centering
    \includegraphics[width=1\textwidth]{figura/recommender_model.eps}
  \caption{Modelo para criação de um sistema de recomendação personalizado (PICAULT et al., 2011)}
\end{figure}
    
\end{frame}

\begin{frame}
    Tipos de sistemas de recomendação:

    \begin{itemize}
        \item Recomendação baseada em conteúdo
        \item Recomendação colaborativa
        \item Recomendação híbrida
        \item Recomendação por contexto
    \end{itemize}
\end{frame}

\begin{frame}

\begin{figure}[h!]
  \centering
    \includegraphics[width=1\textwidth]{figura/recomendacao_conteudo.eps}
  \caption{Fluxo para construção de um sistema de recomendação por conteúdo (LOPS; GEMMIS; SEMERARO, 2011)}
\end{figure}
    
\end{frame}


% section l (end)

\section{Software escolhido} % (fold)
\label{sec:software_escolhido}
\begin{frame}
	\say {\textit{Client should not be forced to depend on methods it does not use}},\\Martin, Robert(2002)
\end{frame}
\begin{frame}
\begin{figure}
        \centering
        \begin{minipage}{.5\textwidth}
            \centering
            \includegraphics[width=.4\linewidth]{ipossivelerro2.png}
            \caption{Model was a abstraction class}
        \end{minipage}%
        \begin{minipage}{.5\textwidth}
            \centering
            \includegraphics[width=.4\linewidth]{improvement2.png}
            \caption{Possible application of ISP}
        \end{minipage}
    \end{figure}
\end{frame}
% section i (end)

\section{Contexto temporal no Debian} % (fold)
\label{sec:d}

\begin{frame}

    \begin{itemize}
        \item Extrair quando pacote foi usado pela última vez
        \item Comando stat
        \item Popularity-contest
    \end{itemize}

\end{frame}

\begin{frame}
\begin{figure}[h!]
    \centering
    \includegraphics[width=1\textwidth]{figura/comando_stat.eps}
    \caption{Saída do comando stat para um dado software}
\end{figure}
\end{frame}

\begin{frame}

Fórmula usada para gerar um valor de contexto temporal para um dado pacote:
\newline
\newline
ClassificaçãoTempo = $\frac{TempoAcesso - TempoModificação}{TempoAtual -
TempoModificação}$

\end{frame}


\begin{frame}
    Restrição quanto ao sistema de arquivo

    \begin{itemize}
        \item noatime
        \item stricatime
        \item relatime
    \end{itemize}
\end{frame}

\begin{frame}

    \begin{itemize}
        \item Pacotes usados sem intervenção direta do usuário
            \begin{itemize}
                \item Retirada de pacotes com prioridade \textit{Required},
                      \textit{Important} e \textit{Standard}.
            \end{itemize}
        \item Flag relatime não apresenta valores exatos
    \end{itemize}

\end{frame}

% section d (end)
\section{Estratégia determinística} % (fold)
\label{sec:our_code}

\begin{frame}

Using SOLID for points of improvement in our code, wih priority to:
\begin{itemize} 
\setlength{\leftmargini}{2.5em}
\item     Single-responsiblity principle
\item     Open-closed principle 
\item     Dependency Inversion Principle
\end{itemize}
\end{frame}

\begin{frame}
\begin{figure}[!htb]
\centering
\includegraphics[scale=0.5]{SOLID_ANALIZO}
\caption{Analizo - Refactoring Clang Extractor}
\end{figure}
\end{frame}

\section{Estratégia determinística} % (fold)
\label{sec:estrategia_deterministica}

\section{Aprendizado de máquina} % (fold)

\begin{frame}

    \begin{itemize}
        \item Estratégia de pós-filtragem
        \item Rótulos do pacote
        \item Formatação dos dados
        \item Estratégia de aprendizado
    \end{itemize}

\end{frame}

\begin{frame}

    Rótulos do pacote:
    \newline
    \newline

    \begin{table}[h]
    \centering
    \begin{tabular}{cc}
    \hline
    \rowcolor[HTML]{EFEFEF}
    {Escala} & {Valores} \\ \hline
    {Excelent(EX)}  & ClassificaçãoTempo >= 0.8                  \\ \hline
    {Great(G)}   & ClassificaçãoTempo >= 0.7                  \\ \hline
    {Medium(M)}   & ClassificaçãoTempo >= 0.5                  \\ \hline
    {Bad(B)}   & ClassificaçãoTempo >= 0.3                  \\ \hline
    {Horrible(H)}   &ClassificaçãoTempo < 0.3                   \\ \hline
    \end{tabular}
    \caption{Escala para classificação de um pacote baseado em seus atributos de tempo}
    \label{tab:classificacao_pacotes}
    \end{table}

\end{frame}

\begin{frame}

    \begin{itemize}
        \item Vetores binários para representar um pacote
        \item Debtags e descrição do pacote
        \item Termos usados apenas dos pacotes manualmente instalados pelo
              usuário
    \end{itemize}
\end{frame}

\begin{frame}

    \begin{itemize}
        \item Bayes ingênuo
        \item Uso da probabilidade de bayes
        \item Distribuição de Bernoulli
        \item Dependência entre variáveis não levada em conta
        \item Linearização do método
    \end{itemize}

\end{frame}

\begin{frame}

    Fórmula adotada para o bayes ingênuo:
    \newline
    \newline
    $Classificador = max(p(C_{y})*\prod_{i=1}^{N}p(x_{i}|C_{y}))$

\end{frame}

\begin{frame}

    \begin{itemize}
        \item Validação cruzada
        \item Curvas de Aprendizado
    \end{itemize}

\end{frame}

\begin{frame}

\begin{figure}[h]
  \centering
  \includegraphics[width=0.9\textwidth]{figura/curva_aprendizado.eps}
  \caption{Exemplo de uma curva de aprendizado}
  \label{fig:curva_aprendizado}
\end{figure}

\end{frame}


\label{sec:aprendizado_maquina}

\section{Comparação dos resultados} % (fold)
\label{sec:comparacao_resultados}

\begin{frame}
    \begin{itemize}
        \item Comparação baseadas no \textit{AppRecommender}
        \item Foco em precisão e novidade das recomendações
        \item Comparação \textit{offline}
        \item Teste com usuário
    \end{itemize}
\end{frame}

\begin{frame}
\begin{figure}[h]
  \centering
  \includegraphics[width=0.7\textwidth]{figura/curva_roc.eps}
  \caption{Exemplo de curva ROC}
  \label{fig:curva_roc}
\end{figure}

\end{frame}

\begin{frame}

    \begin{itemize}
        \item Usuário será apresentado a 5 recomendações de cada vez
        \item 4 estratégias distintas
        \item 20 recomendações de pacotes no total
    \end{itemize}

\end{frame}

\begin{frame}

 \begin{itemize}
    \item \textbf{Ruim: } Recomendação que não agrada ao usuário.
    \item \textbf{Redundante: } Usuário possui aplicativos similares para o item
        sendo recomendado.
    \item \textbf{Útil: } Usuário acha que a recomendação lhe proporciona um
            pacote útil.
    \item \textbf{Surpresa boa: } Usuário considera a recomendação útil e além
        do mais inesperada.
\end{itemize}
   

\end{frame}

\begin{frame}
    
\begin{itemize}
    \item \textbf{Precisão: } $\frac{VerdadeirosPositivos}{VerdadeirosPositivos
        + FalsoNegativos}$
    \item \textbf{Novidade: } $\frac{NumSurpresaBoa}{VerdadeirosPositivos +
        FalsoNegativos}$
\end{itemize}

Onde:

    \begin{itemize}
        \item VerdadeirosPositivos: Útil e Surpresa boa
        \item FalsoNegativos: Ruim e Redundante
    \end{itemize}

\end{frame}

\section{Resultados parciais} % (fold)
\label{sec:resultados_parciais}

\section{Obrigado!}
\label{sec:obrigado}


