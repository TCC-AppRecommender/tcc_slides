\documentclass[xcolor=table]{beamer}

\mode<presentation> {

% The Beamer class comes with a number of default slide themes
% which change the colors and layouts of slides. Below this is a list
% of all the themes, uncomment each in turn to see what they look like.

% \usetheme{default}
% \usetheme{AnnArbor}
\usetheme{Antibes} % nice one
% \usetheme{Bergen}
% \usetheme{Berkeley}
% \usetheme{Berlin}
% \usetheme{Boadilla}                
% \usetheme{CambridgeUS} %kkkkkkkkkkkk
% \usetheme{Copenhagen}
% \usetheme{Darmstadt}  % 22222222222222
% \usetheme{Dresden}
% \usetheme{Frankfurt}
% \usetheme{Goettingen} 
% \usetheme{Hannover} %llllllllllllll
% \usetheme{Ilmenau}
% \usetheme{JuanLesPins} %other nice one
% \usetheme{Luebeck}
% \usetheme{Madrid} %3333333333333333333
% \usetheme{Malmoe}
% \usetheme{Marburg}  %not that bad
% \usetheme{Montpellier} %2222222
% \usetheme{PaloAlto}
% \usetheme{Pittsburgh}
% \usetheme{Rochester}
% \usetheme{Singapore}
% \usetheme{Szeged} % 0000000000000000
% \usetheme{Warsaw}

% As well as themes, the Beamer class has a number of color themes
% for any slide theme. Uncomment each of these in turn to see how it
% changes the colors of your current slide theme.

%\usecolortheme{albatross}
%\usecolortheme{beaver} % 0000000000
%\usecolortheme{beetle}
%\usecolortheme{crane}
\usecolortheme{dolphin} % 0000000
%\usecolortheme{dove}   %%%%%%%%%%
%\usecolortheme{fly}
%\usecolortheme{lily}
%\usecolortheme{orchid} %%%%%%%%%%%
%\usecolortheme{rose}
%\usecolortheme{seagull}
%\usecolortheme{seahorse}
%\usecolortheme{whale}
%\usecolortheme{wolverine}

%\setbeamertemplate{footline} % To remove the footer line in all slides uncomment this line
%\setbeamertemplate{footline}[page number] % To replace the footer line in all slides with a simple slide count uncomment this line

%\setbeamertemplate{navigation symbols}{} % To remove the navigation symbols from the bottom of all slides uncomment this line
}

\usepackage{graphicx} % Allows including images
\usepackage{booktabs} % Allows the use of \toprule, \midrule and \bottomrule in tables
\usepackage[brazil]{babel}
\usepackage[T1]{fontenc}	
\usepackage[utf8]{inputenc}	
\usepackage{verbatim}
\usepackage{xcolor,lipsum}
\usepackage{listings,showexpl}
\usepackage{amsmath}
\usepackage{multicol}
\usepackage{transparent}
\usepackage{epigraph}


\pdfcompresslevel=9
\pgfdeclareimage[width=2.5cm]{logo}{logo}
\logo{\transparent{0.2}\pgfuseimage{logo}}

%--------------------------------------------
%Definições para código com fundo listrado
\newcommand\realnumberstyle[1]{#1}
\makeatletter
\newcommand{\zebra}[3]{%
    {\realnumberstyle{#3}}%
    \begingroup
    \lst@basicstyle
    \ifodd\value{lstnumber}%
        \color{#1}\pgfsetfillopacity{0.9}%
    \else
        \color{#2}%
    \fi
        \rlap{\hspace*{\lst@numbersep}%
        \color@block{\linewidth}{\ht\strutbox}{\dp\strutbox}%
        }%
    \endgroup
}
\makeatother

\lstset{%
language=tex,           %linguagem
numbers=left,           %posição dos números
stepnumber=1,           %frequencia de aparição dos números
numbersep=5pt,
numberstyle=\zebra{black!10}{white!35},
basewidth={0.6em,0.45em},
fontadjust=true,
mathescape=true,
tabsize=4,
commentstyle=\color{blue},
literate={á}{{\'a}}1 {à}{{\`a}}1 {ã}{{\~a}}1 {é}{{\'e}}1 {É}{{\'E}}1 {ê}{{\^e}}1 {õ}{{\~o}}1 {í}{{\'i}}1 {ó}{{\'o}}1 {ú}{{\'u}}1 {ç}{{\c c}}1 {³}{{$^3$}}1 {Ω}{{$\Omega$}}1,
breaklines=true,
showstringspaces=false,
stringstyle=\color{cyan},
basicstyle=\tiny\ttfamily}

\newcommand{\code}{\small\ttfamily}
% \defbeamertemplate{description item}{align left}{\insertdescriptionitem\hfill}
% \setbeamertemplate{description item}[align left]

\usepackage{dirtytalk}
\usepackage{parallel}